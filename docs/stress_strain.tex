\documentclass[12pt]{report}  % include bezier curves
\renewcommand\baselinestretch{1.0}           % single space
%\pagestyle{empty}                            % no headers and page numbers
\oddsidemargin -10 true pt      % Left margin on odd-numbered pages.
\evensidemargin 10 true pt      % Left margin on even-numbered pages.
\marginparwidth 0.75 true in    % Width of marginal notes.
\oddsidemargin  0 true in       % Note that \oddsidemargin=\evensidemargin
\evensidemargin 0 true in
\topmargin 0.25 true in        % Nominal distance from top of page to top of
\textheight 9.0 true in         % Height of text (including footnotes and figures)
\textwidth 6.375 true in        % Width of text line.
\parindent=0pt                  % Do not indent paragraphs
\parskip=0.15 true in
\usepackage{color}              % Need the color package
\usepackage{epsfig}

\usepackage{algorithmic}


\title{NRSF2 Design Document Appendix: Strain and Stress Calculation}

\begin{document}

%\maketitle

{\bf Strain}, aka, unconstrained strain, is measured as the fraction change from a reference state ($d_0$).
\begin{eqnarray}
\epsilon_{ij} &=& \frac{d_{ij} - d_0}{d_0}
\end{eqnarray}

{\bf Residual stress} is determined by measuring stress along {\bf\it 3} orthogonal directions
\begin{eqnarray}
\sigma_{ij} &=&
	\frac{E}{(1 + \nu)}\left[\epsilon_{ij} + \frac{\nu}{1-2\nu}(\epsilon_{11} + \epsilon_{22} + \epsilon_{33})\right]
\end{eqnarray}

where
\begin{itemize}
\item $\nu$ is {\it Poisson's Ratio}.
\item $E$ is {\it Young's Modulus}.
\item $\epsilon_{ij}$ are strains.  Be noted that
	\begin{itemize}
	\item  $\epsilon_{ij}$ with $i = j$ are principle strains.  But not all all three orthogonal strains are equivalent to principle strains.
	\item The off-diagonal strain component, i.e., $\epsilon_{ij}$ with $i\neq j$ are all set to {\bf zero}.  It is very hard to measure these values in HB2B's setup.
	\end{itemize}
\end{itemize}

Therefore the stress that is calculated is
%
\begin{equation}
\sigma_{ii} =
	\frac{E}{(1 + \nu)}\left[\epsilon_{ii} + \frac{\nu}{1-2\nu}(\epsilon_{11} + \epsilon_{22} + \epsilon_{33})\right]
\end{equation}
%
where the second term in the sum is the same between all 3 principle strain directions, ($\sigma_{11}$, $\sigma_{22}$, and $\sigma_{33}$).

There are also two simplified cases when only two strain components are measured. The first is {\bf in-plane strain}, where $\epsilon_{33}=0$. Then the strain equations become
%
\begin{eqnarray}
\sigma_{11} &=&
	\frac{E}{(1 + \nu)}\left[\epsilon_{11} + \frac{\nu}{1-2\nu}(\epsilon_{11} + \epsilon_{22})\right] \\
\sigma_{22} &=&
	\frac{E}{(1 + \nu)}\left[\epsilon_{22} + \frac{\nu}{1-2\nu}(\epsilon_{11} + \epsilon_{22})\right] \\
\sigma_{33} &=&
	\frac{E \nu (\epsilon_{11} + \epsilon_{22})}{(1 + \nu)(1-2\nu)}
\end{eqnarray}

The {\bf in-plane stress}, assumes $\sigma_{33} = 0$.
Therefore, $\epsilon_{33}$ can be calculated from $\epsilon_{11}$ and $\epsilon_{22}$ from $\sigma_{33} = 0$. Then the missing strain can be determined to be
%
\begin{equation}
\epsilon_{33} = \frac{\nu}{\nu-1}(\epsilon_{11} + \epsilon_{22})
\end{equation}
%
With that relation, the stresses (with the in-plane stress assumption) are
%
\begin{eqnarray}
\sigma_{11} &=&
	\frac{E}{(1 + \nu)}\left[\epsilon_{11} + \frac{\nu (\epsilon_{11} + \epsilon_{22})}{1-\nu}\right] \\
\sigma_{22} &=&
	\frac{E}{(1 + \nu)}\left[\epsilon_{22} + \frac{\nu (\epsilon_{11} + \epsilon_{22})}{1-\nu}\right] \\
\sigma_{33} &=& 0
\end{eqnarray}

\end{document}
