\documentclass[12pt,bezier,amstex]{report}  % include bezier curves
\renewcommand\baselinestretch{1.0}           % single space
%\pagestyle{empty}                            % no headers and page numbers
\oddsidemargin -10 true pt      % Left margin on odd-numbered pages.
\evensidemargin 10 true pt      % Left margin on even-numbered pages.
\marginparwidth 0.75 true in    % Width of marginal notes.
\oddsidemargin  0 true in       % Note that \oddsidemargin=\evensidemargin
\evensidemargin 0 true in
\topmargin 0.25 true in        % Nominal distance from top of page to top of
\textheight 9.0 true in         % Height of text (including footnotes and figures)
\textwidth 6.375 true in        % Width of text line.
\parindent=0pt                  % Do not indent paragraphs
\parskip=0.15 true in
\usepackage{color}              % Need the color package
\usepackage{epsfig}

\usepackage{algorithmic}


\title{NRSF2 Design Document Appendix}

\begin{document}

\maketitle

Construct 2 orthogonal {\it q} vectors:

\begin{eqnarray*}
\vec{q}_1 &=& 
	R_z (- \frac{2\theta}{2}) (010)  \\
\vec{q}_2 &=&
	R_z(-\frac{2\theta}{2}) (100)
\end{eqnarray*}

where 
\begin{itemize}
\item	$2\theta_{peak}$ = position of peak (which is in plane)
%\item	$\omega$ = incident angle
%\item	$\chi$ = $\chi$ rotation about (100) of sample
%\item	$\phi$ =  $\phi$ rotation about sample normal
\end{itemize}

In order to convert $\vec{q}$ from instrument coordinate to sample coordinate, the ration will be done on
$\omega$, $\chi$ and $\phi$ respectively. 

The rotation matrix $R_p$ is defined as
\begin{eqnarray*}
R_p &=& R_x(\phi + 90^o) \times R_y(\chi) \times R_z(-\omega)
\end{eqnarray*}

where 
\begin{itemize}
\item	$\omega$ = incident angle
\item	$\chi$ = $\chi$ rotation about (100) of sample
\item	$\phi$ =  $\phi$ rotation about sample normal
\end{itemize}

Thus, $\vec{q}_1$ and $\vec{q}_2$ are rotated as
% TO BE CONTINUED


The projection is then
\begin{eqnarray*}
\alpha
	&=& \cos^{-1}(Q) (001))	\\
\beta
	&=& \cos^{-1}(Q) (100))	\\
\alpha
	&=& \cos^{-1}(Q\prime) (001))	\\
\beta
	&=& \cos^{-1}(Q\prime) (100))	\\
\alpha_p
	&=& \alpha - \alpha\prime	\\
\beta_p
	&=& \beta - \beta\prime
\end{eqnarray*}

\end{document}

